\documentclass[12pt]{article}
\usepackage{amsmath}
\usepackage{amssymb}
\usepackage{geometry}
\usepackage{xcolor}
\usepackage{tcolorbox}
\usepackage{enumitem}

\geometry{margin=1in}
\tcbuselibrary{skins,breakable}

\definecolor{primaryblue}{RGB}{102, 126, 234}
\definecolor{accentpurple}{RGB}{118, 75, 162}

\title{\textbf{\Large Polynomial Ensemble for Stock Prediction}}
\author{}
\date{}

\begin{document}

\maketitle

\begin{tcolorbox}[colback=primaryblue!5,colframe=primaryblue!80!black,title=\textbf{Definition},fonttitle=\bfseries\large]
$f(x)$ is a set of polynomials for a specific stock.
\end{tcolorbox}

\vspace{1em}

\begin{tcolorbox}[colback=accentpurple!5,colframe=accentpurple!80!black,title=\textbf{Set Description},fonttitle=\bfseries\large,breakable]
This set can be described as:
\begin{equation*}
\left\{ f(x) \mid f(x) = ax^5 + bx^4 + cx^3 + dx^2 + ex + f \right\}
\end{equation*}

\vspace{0.5em}

where:
\begin{itemize}[leftmargin=2em]
    \item Each coefficient $(a, b, c, d, e, f)$ is unique to each polynomial, and determined through a polynomial regression of a random 5 points.
\end{itemize}
\end{tcolorbox}

\vspace{1em}

\begin{tcolorbox}[colback=green!5,colframe=green!60!black,title=\textbf{Ensemble Bundle},fonttitle=\bfseries\large,breakable]
Take many $f(x)$'s like this ($N$ of them) and create a bundle:

\begin{equation*}
\mathcal{F} = \left\{ f_1(x), f_2(x), f_3(x), \ldots, f_N(x) \right\}
\end{equation*}

where each $f_i(x) = a_i x^5 + b_i x^4 + c_i x^3 + d_i x^2 + e_i x + f_i$
\end{tcolorbox}

\vspace{1em}

\begin{tcolorbox}[colback=orange!5,colframe=orange!70!black,title=\textbf{Ensemble Functions},fonttitle=\bfseries\large,breakable]

The bundle $\mathcal{F}$ has the following functions:

\vspace{0.5em}

\textbf{1. rankBuyOrSell(buyDate, sellDate)}

\vspace{0.3em}

This function returns how many of the $f(x)$'s return positive change, and how many return negative change:

\begin{align*}
\text{positive} &= \left| \left\{ f_i \in \mathcal{F} \mid f_i(\text{sellDate}) - f_i(\text{buyDate}) > 0 \right\} \right| \\[0.5em]
\text{negative} &= \left| \left\{ f_i \in \mathcal{F} \mid f_i(\text{sellDate}) - f_i(\text{buyDate}) < 0 \right\} \right|
\end{align*}

\textbf{Returns:} $(\text{positive}, \text{negative})$

\vspace{1em}

\textbf{2. average()}

\vspace{0.3em}

This function returns the average function:

\begin{equation*}
h(x) = \frac{1}{N} \sum_{i=1}^{N} f_i(x)
\end{equation*}

Expanded form:
\begin{equation*}
h(x) = \frac{f_1(x) + f_2(x) + f_3(x) + \cdots + f_N(x)}{N}
\end{equation*}

\textbf{Returns:} A polynomial function $h(x)$ of degree 5

\end{tcolorbox}

\vspace{1em}

\begin{tcolorbox}[colback=blue!5,colframe=blue!60!black,title=\textbf{Example Usage},fonttitle=\bfseries\large]
Given $N = 100$ polynomials for Stock XYZ:

\begin{itemize}[leftmargin=2em]
    \item $\mathcal{F} = \{f_1(x), f_2(x), f_3(x), f_4(x), f_5(x)\}$
    \item \texttt{rankBuyOrSell(Jan 1 1999, Feb 12 2004)} might return: (positive: 70, negative: 30)
    \item \texttt{average()} returns: $h(x) = \frac{1}{100}\sum_{i=1}^{100} f_i(x)$
\end{itemize}

Challenging for a human to do by-hand, but not so challenging for a computer.
Nothing in this algorithm expands beyond O(n) complexity.

\end{tcolorbox}

\end{document}